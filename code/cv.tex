\documentclass [10pt, letterpaper]{moderncv}
%\moderncvstyle{classic}
%\usepackage{html}
%\usepackage{hyperref}
\moderncvtheme[blue]{classic}
%\moderncvtheme[blue]{casual}
\usepackage [utf8]{inputenc}
\usepackage [scale=0.8]{geometry}
\usepackage{url}
\AtBeginDocument {\recomputelengths}

\firstname{J. Perry}
\familyname{Evans}
%\address{2530 Market Street, Suite 1067}{Philadelphia, PA, 19104}
\email{evansj@email.chop.edu}
%\phone{XXX XXX XXXX}
%\extrainfo{\weblink{http://hostpathogen.blogspot.com/}}
%\extrainfo{\htmladdnormallink{hostpathogen.blogspot.com}{http://hostpathogen.blogspot.com/}}

\newcommand\myitem{\item[ $\circ$ ]\hspace*{.1em} }

\begin {document}
\maketitle

%\def\ConTeXt{
%  C
%  \kern-.0333emo
%  \kern-.0333emn
%  \kern-0.667em\TeX
%  \kern-0.333emt}

\section {Career Summary}
\cvline {} {In my current role at the Children's Hospital of Philadelphia, I provide computational support for research as part of the bioinformatics core. My projects have included an analysis of germline pediatric variants, characterization of cysteine posttranslational modifications, operations research for blood transfusions, and an omics study of mitotic exit.}

\section {Education}

%% \cventry {2010--2013}
%%          {Postdoctoral Fellow}
%%          {Yale Center for Medical Informatics; Yale University}
%%          {New Haven, CT}
%%          {}
%%          {}

\cventry {2005--2010}
         {PhD in Genomics and Computational Biology}
         {University of Pennsylvania}
         {Philadelphia, PA}
         {}
         {}

\cventry {2001--2005}
         {BS in Computer Science}
         {Rose--Hulman Institute of Technology}
         {Terre Haute, IN}
         {}
         {}

\section {Experience}

\cventry {2013--Current}
         {Bioinformatics Scientist}
         {Department of Biomedical and Health Informatics}
         {Children's Hospital of Philadelphia}
         {Philadelphia, PA}
         {I  provide computational consultation and support in collaborations with  hospital investigators
           \begin{itemize}%
             \setlength{\itemindent}{1.5em}
         \myitem Analysis of variants predisposing to pediatric cancer using 600 whole genomes
         \myitem Structural and functional analysis of cysteine modifications
         \myitem ChIP-Seq and RNA-Seq analysis for post mitotic reboot
         \myitem Calling and analysis of variants from ENCODE ChIP-Seq data
         \myitem Analysis of PRO-seq data
         \myitem Variant prioritization for cancer and epilepsy patients
           \end{itemize}
           }

\cventry {2010--2013}
         {Postdoctoral Fellow}
         {Yale Center for Medical Informatics,
          Yale University}
         {New Haven, CT}
         {}
         {Worked closely with melanoma biologists to analyze exome sequence data from over 200 paired normal and tumor samples to discover new genes important for melanoma
           \begin{itemize}%
             \setlength{\itemindent}{1.5em}
         \myitem Developed a pipeline to call novel somatic single nucleotide variations (SNVs) from exome sequences
         \myitem Integrated public resources such as TCGA, COSMIC, and 1000 Genomes to evaluate SNV function
         \myitem Developed a new method to identify common focal copy number variations (CNVs) across tumors
         \myitem Developed novel tools to identify genes with a significant number of nonsynonymous mutations
         \myitem Integrated expression, CNVs, SNVs, and indels to arrive at list of genes important for melanoma
           \end{itemize}
           }

%% Michael Krauthammer's team leads the ongoing bioinformatics analysis of over 150 paired 
%%           tumor/normal exomes sequenced as part of the Yale SPORE in Skin Cancer project.
%%           Under his guidance, I developed new tools and methods to utilize exome sequence data to identify
%%           novel somatic mutations in tumors, detect recurrent tumor copy number aberrations,
%%           and find genes with a significant number of nonsynonymous somatic mutations. 
%%           My work helped identity a list of candidate genes that might drive melanoma.

\cventry {2012--2013}
         {Consultant}
         {FXI R\&D}
         {Aston, PA}
         {}
         {Developed a database and website for FXI, a polyurethane foam company
           \begin{itemize}
             \setlength{\itemindent}{1.5em}
             \myitem Developed an automated web-crawler to gather data and store it in a MySQL database
             \myitem Built an interactive website for researchers to retrieve, monitor, and plot data
             \myitem Guided an intern through Access Database construction and its interface with Excel
             \end{itemize}
         }

\cventry {2006--2010}
         {Research Assistant}
         {Computer and Information Science Department, 
          University of Pennsylvania}
         {Philadelphia, PA}
         {}
         {Conducted dissertation research on biological networks under the direction of Lyle Ungar 
           \begin{itemize}
             \setlength{\itemindent}{1.5em}
           \myitem Developed a new approach to align biological networks
           \myitem Developed a method to predict virus-host protein interactions  based on protein and network motifs
           \end{itemize}
         }

\cventry {2007--2010}
         {Research Assistant}
         {Center for Integrated Bioinformatics at Drexel University}
         {Philadelphia, PA}
         {}         
         {Assisted professor Aydin Tozeren in running the
          bioinformatics lab at Drexel
          \begin{itemize}
           \setlength{\itemindent}{1.5em}
\myitem Helped write the narrative and preliminary studies section for multiple NIH grants
\myitem Facilitated communication between labs at Drexel, UPenn, and George Washington University
\myitem Helped administrate computational resources, including a 56 node IBM Bladecenter
\end{itemize}
}

\cventry{2008--2010}
        {Teaching Assistant}
        {School of Biomedical Engineering, Drexel University}
        {Philadelphia, PA}
        {}
        {Aided professor Aydin Tozeren in teaching his Quantitative Systems Biology graduate class
         \begin{itemize}
           \setlength{\itemindent}{1.5em}
          \myitem Gave lectures describing the methodology behind my research
           \myitem Guided students through computational labs I designed
         \myitem Mentored students as they completed their final projects
        \end{itemize}
        }

\cventry{August, 2008}
        {Instructor}
        {Greater Philadelphia Bioinformatics Alliance}
        {Philadelphia, PA} {}
        {Organized and taught a two week online introduction to Biopython
        \begin{itemize}%
          \setlength{\itemindent}{1.5em}
        \myitem Developed course curriculum to cover sequence annotation, motif discovery, and database integration
        \myitem Guided students through the development of Python scripts for bioinformatics analyses
        \myitem Answered questions via email and graded assignments
        \end{itemize} 
        }

%% \cventry{August, 2008}
%%         {Instructor}
%%         {GPBA Bioinformatics Programming Short Course}
%%         {Philadelphia, PA}
%%         {}
%%         {The Greater Philadelphia Bioinformatics Alliance
%%          is a consortium of regional research and academic
%%          organizations dedicated to advancing bioinformatics.
%%          I was the instructor for a two week online session
%%          that guided participants through the development
%%          of Python scripts for bioinformatics analysis.
%%          The session included an introduction to Biopython
%%          and focused on sequence annotation and function
%%          prediction, the application of Markov models for
%%          gene feature prediction, motif discovery, and 
%%          database integration. I developed the curriculum,
%%          answered questions via email, and graded
%%          assignments.}

%% \cventry {Spring Semester, 2006} 
%%          {Research Assistant} 
%%          {Genetics
%%   Department, University of Pennsylvania} 
%%          {Philadelphia, PA} 
%%          {} 
%%          {The Hannenhalli Lab is interested in the regulation of transcription 
%%           and translation.  During my research rotation, I studied co--evolution
%%           in transcription factor binding sites (TFBSs). For each position in
%%           a TFBS, I constructed phylogenetic trees from five mammalian species.
%%           I compared trees at different positions and investigated different 
%%           metrics for capturing dependencies between the
%%           nucleotide identities at these positions.}

%% \cventry {Fall Semester, 2005}
%%          {Research Assistant}
%%          {Wistar Institute}
%%          {Philadelphia, PA}
%%          {}
%%          {The Maley Lab at the Wistar Institute studies the evolution of cancer
%%           using both \emph{in vitro} and \emph{in silico} techniques.  
%%           While in the
%%           lab, I studied selective interference in yeast with the goal of
%%           applying my findings to cancer treatment.  I designed experiments
%%           to observe the effects of selective interference in yeast under
%%           three selective pressures. I developed agent--based models to
%%           describe the results observed in my experiments.}

%% I am seeking a position where I can build on
%%   my experience with exome sequence analysis.

%and participating in
%  the design of `wetlab' experiments.

  %% seek a computational biology postdoctoral position where I can
  %% explore new datasets and participate in `wetlab' experiments.

%% I
%%   wish to expand my work with virus--host interactions by examining
%%   the evolution of viruses in response to host selective pressures.

%% with human diseases by exploring and
%%   integrating other high--dimensional datasets and participating in
%%   the design of `wetlab' experiments.

% virus--host interactions by including structural information in interaction prediction, learning how virus strain diversity affects interactions, and participating in `wetlab' experiments.

        %formingcollaborations with experimentalists who can test my hypotheses \textit{in vivo}.

\section {Technical Skills}

\subsection{Extremely Proficient With}
\cvline{languages}{Python, R/Bioconductor, JavaScript D3, Bash Scripting}
\cvline{technologies}{Linux, OS X, MySQL, Git, Snakemake, \LaTeX{}, Sun Grid Engine}
\cvline{datatypes}{Next-gen sequencing (WXS, WGS, ChIP-Seq, RNA-Seq), Protein/gene networks, Protein sequence and structure data}
\cvline{mathematics}{Development and application of statistical and machine learning methods to problems in bioinformatics/comp bio, structural biology, systems biology, biostatistics, epidemiology, genetics}

\subsection{Have Experience With}
\cvline{languages}{Java, Go, PHP, JavaScript, Matlab, HTML, CSS, Ruby, C, C++, C\#, Scheme}
\cvline{technologies}{Windows, TORQUE batch queuing system, JSON, Subversion}
\cvline{datatypes}{CaptureC, Peptide counts, DNase hypersensitivity}

% Publications from a BibTeX file
\nocite{*}
%ieeetr
\bibliographystyle{ieeetr}
%\bibliographystyle{unsrt}
\bibliography{publications}

%% \section{Scientific Publications}
%% {
%%     \renewcommand{\section}[2]{\subsection{#2}}
%%     \setbiblabelwidth{99}
%%     \bibliographystyle{ieeetr}
%%     \bibliography{publications}
%% }

\section {\href{http://bit.ly/9F9a08}{Talks and Posters}}

\cventry {April, 2017}
         {A guide to filtering TARGET Complete Genomics germline variants}
         {American Association for Cancer Research Annual Meeting}
         {Washington, DC}
         {}
         {My poster described a decision tree for classifying false positive variant calls in Complete Genomics data.}

\cventry {October, 2012}
         {Estimating a gene's mutation burden by the number of synonymous base substitutions}
         {IEEE International Conference on Bioinformatics and 
          Biomedicine}
         {Philadelphia, PA}
         {}
         {My talk compared methods for determining genes with significant nonsynonymous 
          mutation burden based on the ratio of nonsynonymous to synonymous mutation frequencies.}

\cventry {June, 2012}
         {Evaluating melanoma whole exome sequences suggests new gene drivers}
         {National Library of Medicine Informatics Training Conference 2012}
         {Madison, WI}
         {}
         {My poster outlined novel methods for determining genes with significant mutation 
          burden by examining the ratio of nonsynonymous to synonymous mutation frequencies.}

\cventry {October, 2011}
         {Exploring the use of social media to measure journal article impact}
         {American Medical Informatics Association Annual Symposium}
         {Washington DC}
         {}
         {My talk demonstrated that high impact journal articles are cited in Wikipedia pages, 
          while low impact journal articles are not. I suggested that monitoring Wikipedia science 
          updates would help one keep up with the latest science advances.}

\cventry {March, 2010}
         {Modularity in protein interaction network hubs predicts
          viral host--pathogen interactions}
         {Keystone Symposia, Biomolecular Interaction Networks: Function
          and Disease}
         {Qu\`{e}bec City, QC, Canada}
         {}
         {My poster demonstrated that human hub proteins 
          interacting with HIV and HCV
          have a preference between intra/intermodular hubs.}

\cventry {November, 2009}
         {MAPK docking motifs on HIV proteins}
         {Greater Philadelphia Bioinformatics Alliance Annual Research Retreat}
         {Philadelphia, PA}
         {}
         {My talk showed that standard MAPK docking motifs are
          missing from 4 of 5 HIV proteins phosphorylated by ERK1/2, 
          and suggested evidence
          for alternative docking sites.}

\cventry {November, 2008}
         {Protein--protein interaction network alignment by quantitative
          simulation}
         {IEEE International Conference on Bioinformatics and 
          Biomedicine}
         {Philadelphia, PA}
         {}
         {My talk focused on the application of the 
          quantitative simulation algorithm to the 
          alignment of fly and yeast
          protein networks.}

\cventry {September, 2006} 
         {Conservation patterns in
  \emph{cis}--elements reveal co--evolution} 
         {RECOMB Comparative
  Genomics International Workshop} 
         {Montr\`{e}al, QC, Canada} 
         {} 
         {My talk covered a project in which I utilized phylogenetic trees made
          using five
          mammalian species to present evidence for dependencies between
          positions in JASPAR transcription factor binding sites.}



\section {Awards}
\cventry {June, 2012}
         {Best Poster: Evaluating melanoma whole exome sequences suggests new gene drivers}
         {National Library of Medicine Informatics Training Conference 2012}
         {}
         {}
         {Awards were chosen by vote of the conference attendees each day.}
\cventry {2010--2013}
         {NLM Postdoctoral Fellow}
         {National Library of Medicine Training Grant
          in Biomedical Informatics}
         {}
         {}
         {Grant awarded by Yale University.}
\cventry {2007--2010}
         {NIH Training Grant}
         {National Human Genome Research Institute Training Grant
          in Computational Genomics}
         {}
         {}
         {Grant awarded by the University of Pennsylvania.}

%% \pagebreak
%% \section {Professional Activities}

%% \cventry {2007--Current}
%%          {Reviewer}
%%          {}
%%          {}
%%          {}
%%          {I have independently reviewed articles for Virology Journal and BMC Systems Biology. 
%%           Under the supervision of Aydin Tozeren at Drexel University, I have reviewed NIH grants
%%           and articles from BMC Bioinformatics, Frontiers in Bioscience, BMC Genomics, and Human Genomics.}

%% \cventry {2007--Current}
%%          %{\weblink{http://hostpathogen.blogspot.com/}}
%%          {Open Notebook Science}
%%          {\htmladdnormallink{Notebook Link: http://hostpathogen.blogspot.com/}{http://hostpathogen.blogspot.com/}}
%%          {}
%%          {}
%%          {I promote open access science by maintaining a blog
%%           and online computational biology resource.  The blog
%%           serves as my lab notebook, and has accumulated over 
%%           700 posts.}

%% \section {References}
%% \cventry {}{Available on request}{}{}{}{}
%% \cventry {Postdoctoral advisor}
%%          {Michael Krauthammer}
%%          {Associate Professor of Pathology}
%%          {Yale University, New Haven, CT}
%%          {\htmladdnormallink{Website}{http://www.yalepath.org/research_labs/krauthammer/index.htm}}
%%          {michael.krauthammer@yale.edu}

%% \cventry {Thesis advisor}
%%          {Lyle Ungar}
%%          {Associate Professor of Computer and Information Science}
%%          {University of Pennsylvania, Philadelphia, PA}
%%          {\htmladdnormallink{Website}{http://www.cis.upenn.edu/~ungar/}}
%%          {ungar@cis.upenn.edu}

%% \cventry {Thesis co-advisor}
%%          {Aydin Tozeren} 
%%          {Distinguished Professor of Biomedical Engineering, Center for Integrated Bioinformatics, School of Biomedical Engineering, Science and Health Systems} 
%%          {Drexel University, Philadelphia, PA}
%%          {\htmladdnormallink{Website}{http://bioinformatics.biomed.drexel.edu/}}         
%%          {aydin.tozeren@drexel.edu}
         %Distinguished Professor and Director

%\cventry {Rotation advisor}
%         {Sridhar Hannenhalli}
%         {Associate Professor of Genetics}
%         {University of Pennsylvania, Philadelphia, PA}
%         {\htmladdnormallink{Website}{http://cagr.pcbi.upenn.edu/}}
%         {sridharh@pcbi.upenn.edu}


\end {document}

%% Under Dr. Tozeren's supervision, 
%%           I have reviewed NIH grants
%%           and articles from BMC Bioinformatics, Frontiers in Bioscience, 
%%           BMC Genomics, and Human Genomics.
